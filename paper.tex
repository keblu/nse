\documentclass[11pt]{article}
\usepackage[round]{natbib}
\usepackage[margin=1in]{geometry}
\geometry{letterpaper}
\usepackage{url}

\begin{document}
\title{nse: Computation of Numerical Standard Errors in R}
\author{David Ardia\\
Institute of Financial Analysis, University of Neuch\^atel, Switzerland\\
D\'epartement de Finance, Assurance et Immobilier, Universit\'e Laval, Qu\'ebec, Canada
\and
Keven Bluteau\\
Institute of Financial Analysis, University of Neuch\^atel, Switzerland\\
}
%\date{}
	
\maketitle

\section*{Summary}

nse is an R package \citep{R} for computing the numerical standard errors, an estimate of the
standard deviation of a simulation result, if the simulation experiment were to be repeated
many times. The package currently implements more than thirty estimators, including 
batch means estimators \citep[][Section~3.2]{Geyer1992}, initial sequence estimators \citet[][Equation~3.3]{Geyer1992}, spectrum at zero estimators \citep{HeidelbergerWelch1981,FlegalJones2010}, heteroskedasticity and autocorrelation 
consistent (HAC) kernel estimators \citep{NeweyWest1987,Andrews1991,AndrewsMonahan1992,NeweyWest1994,Hirukawa2010}, and bootstrap estimators \citet{PolitisRomano1992,PolitisRomano1994,PolitisWhite2004}. 
A simulation study relying on the package is proposed in \citet{ArdiaEtAl2016}. The latest version of the package is available at \url{https://github.com/keblu/nse}.

\bibliographystyle{plainnat}
\bibliography{paper}
	
\end{document}
